\documentclass[letterpaper,10pt]{article}

\usepackage[utf8]{inputenc}
\usepackage[T2A]{fontenc}

\usepackage{makecell}
\usepackage[link=off]{phonenumbers}

\usepackage{latexsym}
\usepackage[empty]{fullpage}
\usepackage{titlesec}
\usepackage{marvosym}
\usepackage[usenames,dvipsnames]{color}
\usepackage{verbatim}
\usepackage{enumitem}
\usepackage[pdftex]{hyperref}
\usepackage{fancyhdr}



\pagestyle{fancy}
\fancyhf{} % clear all header and footer fields
\fancyfoot{}
\renewcommand{\headrulewidth}{0pt}
\renewcommand{\footrulewidth}{0pt}
\usepackage[margin=0.3in]{geometry}
% Adjust margins
\addtolength{\oddsidemargin}{-0.0in}
\addtolength{\evensidemargin}{-0.0in}
\addtolength{\textwidth}{0in}
\addtolength{\topmargin}{1pt}
\addtolength{\textheight}{0.0in}

\urlstyle{same}

\usepackage{xcolor}% http://ctan.org/pkg/xcolor
\usepackage{hyperref}% http://ctan.org/pkg/hyperref
\hypersetup{
  colorlinks=true,
  linkcolor=blue!50!red,
  linkbordercolor=red,
  urlcolor=blue!70!black
}

\raggedbottom
\raggedright
\setlength{\tabcolsep}{0in}

% Sections formatting
\titleformat{\section}{
  \vspace{-10pt}\scshape\raggedright\large
}{}{0em}{}[\color{black}\titlerule \vspace{-7pt}]

%-------------------------
% Custom commands
\def \ifempty#1{\def\temp{#1} \ifx\temp\empty }

\newcommand{\resumeItem}[2]{
  \item\small{
  	\ifempty{#1}#2\else\textbf{#1}{: #2 \vspace{-2pt}}\fi
  }
}

\usepackage[dvipsnames]{xcolor}
\definecolor{mygray}{gray}{0}
\usepackage{fancybox}

\usepackage{lmodern}
\usepackage{tikz}

% Style definition
\tikzset{rndblock/.style={rounded corners,rectangle,draw,outer sep=0pt}}

% Command Definition
% 1 optional to customize the aspect, 2 mandatory: text to be framed
\newcommand{\tframed}[2][]{\tikz[baseline=(h.base)]\node[rndblock,#1] (h) {\color{black}{#2}};}

\newcommand*{\mystrut}{\rule[-0.2\baselineskip]{0pt}{0.8\baselineskip}}
\newcommand{\skill}[1]{\tframed[lightgray]{\mystrut#1}}


\newcommand{\resumeSubheading}[4]{
  \vspace{-1pt}\item
    \begin{tabular*}{0.97\textwidth}{l@{\extracolsep{\fill}}r}
      \textbf{#1} & \textcolor{mygray}{#2} \\
      \textit{\small#3} & \textcolor{mygray}{\textit{\small #4}} \\
    \end{tabular*}\vspace{-5pt}
}

\newcommand{\resumeExpSubheading}[5]{
  \vspace{-1pt}\item
    \begin{tabular*}{0.97\textwidth}{l@{\extracolsep{\fill}}r}
      \textbf{#1}  & \textcolor{mygray}{#2} \\
      \textit{\small#3} & \textcolor{mygray}{\textit{\small #4}} \\
      {\scriptsize#5}
    \end{tabular*}\vspace{4pt}
}

\newcommand{\resumeProjSubheading}[4]{
  \vspace{-1pt}\item
    \begin{tabular*}{0.97\textwidth}{l@{\extracolsep{\fill}}r}
      \textbf{#1}  & \textcolor{mygray}{#2} \\
      \scriptsize {#3} & \textcolor{mygray}{\textit{\small #4}} \\
    \end{tabular*}\vspace{4pt}
}

\newcommand{\resumeSubItem}[2]{\resumeItem{#1}{#2}\vspace{-4pt}}

\renewcommand{\labelitemii}{$\circ$}

\newcommand{\resumeSubHeadingListStart}{\begin{itemize}[leftmargin=*]}
\newcommand{\resumeSubHeadingListEnd}{\end{itemize}}
\newcommand{\resumeItemListStart}{\begin{itemize}[leftmargin=0.2in]}
\newcommand{\resumeItemListEnd}{\end{itemize}\vspace{-5pt}}

\usepackage{changepage}
\newcommand{\resumeDesc}[1]{\begin{adjustwidth}{5pt}{0pt}\vspace{-2pt}{\small{#1}}\end{adjustwidth}}

%-------------------------------------------
%%%%%%  CV STARTS HERE  %%%%%%%%%%%%%%%%%%%%%%%%%%%%


\begin{document}

%----------HEADING-----------------
\begin{center}
    {\huge \scshape Знай Артемий} \\ \vspace{1pt}
    Moscow, Russia \\ \vspace{2pt}
    \small \raisebox{-0.1\height}\faPhone\ +7 (926) 729-18-07 ~ \href{artem.znay@mail.ru}{\raisebox{-0.2\height}\faEnvelope\  \underline{artem.znay@mail.ru}} ~ 
    \href{https://github.com/znako}{\raisebox{-0.2\height}\faGithub\ \underline{https://github.com/znako}}
    \vspace{-8pt}
\end{center}


%-----------EDUCATION-----------------
\section{Образование}
  \resumeSubHeadingListStart
    \resumeSubheading
       {Москвоский Авиационный Институт}{Москва, Россия}
      {Бакалавриат Институт № 8 «Компьютерные науки и прикладная математика»}{09.2021 - настоящее время}
      \\ \vspace{4pt}
      {Кафедра 806 «Вычислительная математика и программирование»}
      \\ \vspace{4pt}
  \resumeSubHeadingListEnd

%-----------WORK-----------------
\section{Опыт работы}
  \resumeSubHeadingListStart
    \resumeSubheading
       {Администрация города Королев}{Королев, Россия}
      {Разработка IT продуктов для автоматизации сбора и обработки информации }{06.2023 - 10.2023}
      \\ \vspace{4pt}
      {на языке Python с использованием библиотеки pandas и Telegram API}
      \\ \vspace{4pt}
  \resumeSubHeadingListEnd
   \resumeSubHeadingListStart
    \resumeSubheading
       {АО "НПО Эшелон"}{Москва, Россия}
      {Специалист. Центр программных разработок}{02.2024 - 03.2024}
      \\ \vspace{4pt}
      {Разработка frontend части админского веб-приложения для управления }
      \\ \vspace{0pt}
      {сетевым маршрутизатором "Рубикон"}
      \\ \vspace{1pt}
      {Стек: TS, SCSS, React, Axios, Redux Toolkit, Vite, Material UI, FSD}
      \\ \vspace{4pt}
  \resumeSubHeadingListEnd
   \resumeSubHeadingListStart
    \resumeSubheading
       {ООО "Яндекс.Технологии"}{Москва, Россия}
      {Стажер-разработчик }{03.2024 - н.в}
      \\ \vspace{4pt}
      {Разработка frontend части веб-приложения ресторанного продукта Яндекс.Еда Вендор}
      \\ \vspace{4pt}
  \resumeSubHeadingListEnd

%-----------SelfCharcterisitc--------
\section{О себе}
    Меня можно характеризовать как ответственного и трудолюбивого человека. Я люблю программирование и занимаюсь им со средней школы. В 10-11 классе посещал курсы по web-разработке, изучал JS, jQuery. После школы активно самостоятельно изучал JS: читал техническую литературу, проходил как бесплатные, так и платные курсы и писал проекты. Успешно окончил обучение в школе программистов HeadHunter по направлению frontend разработка, пройдя лекционную часть с заданиями и защитив командный проект. Учитывая школьный и вузовский опыт, можно сказать, что я хорошо знаю алгоритмы и структуры данных. Так же изучаю и другие ЯП: C++, Python. Готов учиться новому, изучать новые технологии и применять их в проектах.

%-----------Projects-----------------
\section{Проекты}
  \resumeSubHeadingListStart
      \resumeProjSubheading
      {\href{https://github.com/znako/SimpleBank}{\underline{Сайт-лендинг банка}}}{12.2021}
      {\skill{HTML} \skill{CSS} \skill{JS} }
          \resumeDesc{Создал landing page. Продумал UI и для его реализации использовал observer API, lazy loading для оптимизации загрузки фотографий, slider.}
        
      % \resumeProjSubheading
      % {\href{https://github.com/znako/DigitalLibrary}{\underline{Электронный учет книг}}}{04.2022}
      % { \skill{HTML} \skill{CSS}}
      %     \resumeDesc{Web-приложение для электронного учета книг. Многостраничный сайт. Работа в команде с backend-разработчиком.}
      
      % \resumeProjSubheading
      % {\href{https://github.com/znako/PBN}{\underline{Сайт помощи в проектной деятельности}}}{09.2022}
      % {\skill{HTML} \skill{CSS} \skill{PicoCSS}}
      %     \resumeDesc{Разработанный по ТЗ сайт для помощи школьникам в проектной деятельности. На сайте есть главная страница, личный кабинет, страница для создания и контроля задач, регистрация и вход. Вся стилизация сделана на основе фреймворка picoCSS. Работа в команде с backend-разработчиком.}

      %  \resumeProjSubheading
      % {\href{https://github.com/znako/Cardio-App}{\underline{Приложение для учета тренировок}}}{03.2023}
      % {\skill{HTML} \skill{CSS} \skill{JS}}
      %     \resumeDesc{Web-приложение "Кардио" для учета тренировок. Проект базируется на ООП. Была использована библиотека leaflet для реализации взаимодействия пользователя с картами. Взаимодействие с geolocation API, local storage.}
      \resumeProjSubheading
      {\href{https://github.com/znako/React-Quiz}{\underline{Quiz}}}{05.2023}
      {\skill{React} \skill{redux} \skill{react-router} \skill{axios} \skill{Firebase}}
          \resumeDesc{Проект представляет из себя веб-приложение для создания и прохождения квизов/викторин. В приложении также имплементированы регистрация и вход. Управление состоянием приложения производится с помощью redux. Во всех формах осуществляется валидация данных. В качестве базы данных и хостинг платформы был выбран Firebase.}
    \resumeProjSubheading
      {\href{https://github.com/znako/SlavicMarket/tree/main}{\underline{Slavic Store}}}{12.2023}
      {\skill{React} \skill{redux-toolkit} \skill{react-router} \skill{axios} \skill{eslint}}
          \resumeDesc{Учебный проект: "Сервис доставки и отслеживания заказов". Веб-приложение, в котором реализованы 4 роли:
          \begin{itemize}
          \item Клиент
          \begin{itemize}
            \item Отображение всех товаров
            \item Корзина
            \item Профиль для отслеживания заказов
          \end{itemize}
          \item Курьер
          \begin{itemize}
            \item Отображение всех свободных заказов
            \item Возможность взять заказ и завершить его
            \item Профиль для отслеживания своих заказов
          \end{itemize}
          \item Магазин
          \begin{itemize}
            \item Возможность добавить товар на сайт
            \item Профиль для отображения всех товаров магазина
          \end{itemize}
          \item Администратор
          \begin{itemize}
            \item Удаление товаров
          \end{itemize}
          \end{itemize}
        Архитектура frontend-части приложения основана на методологии Feature-Sliced}.
        Проект выполнялся в команде, состоящей из teamlead'a, двух backend-разработчиков и frontend-разработчика (меня).

  \resumeSubHeadingListEnd

%--------PROGRAMMING SKILLS------------
\section{Hard Skills}
 \resumeSubHeadingListStart
   \item{
     \textbf{ \textbf{HTML, CSS, JS, TS, React} }
   }\vspace{-7pt}
   \item{
     \textbf{Технологии: \textbf{Redux (toolkit), Jest, Storybook, Webpack, CI/CD, git} }
   }\vspace{-7pt}
   \item{
     \textbf{Знания}{: Алгоритмы и структуры данных, Операционные системы}
   }
 \resumeSubHeadingListEnd
 
\section{Soft Skills}
 \resumeSubHeadingListStart
   \item{
     \textbf{Обучаемость, адаптивность, знание английского языка на уровне чтения документации, работа в команде}
    }
 \resumeSubHeadingListEnd

%-------------------------------------------
\end{document}